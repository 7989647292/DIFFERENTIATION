\documentclass{article}
\usepackage{amsmath}
\usepackage{amssymb}
\usepackage{amsfonts}
\usepackage{titling}
\usepackage{graphicx}


\providecommand{\abs}[1]{\left\vert#1\right\vert}
\let\vec\mathbf

\begin{document}

\title{\textbf{DIFFERENTIATION}}
\date{}
\maketitle

\begin{enumerate}
\item The point at which the normal to the curve $ y = x + \frac{1}{x}, X > 0$ is perpendicular to the line $3x - 4y 7 = 0$ is:
	\begin{enumerate}
	\item $(2, \frac{5}{2})$
	\item $(\pm2, \frac{5}{2})$
      	\item $(-\frac{1}{2}, \frac{5}{2})$
      	\item $(\frac{1}{2}, \frac{5}{2})$
  	\end{enumerate}

\item If $y = log(\cos e^x)$, then $\frac{dx}{dy}$ is: 
   
	\begin{enumerate}
	\item $ \cos e^{x-1} $
	\item $ e^{-x} \cos e^x $
      	\item $ e^x \sin e^x $
      	\item $ -e^x \tan e^x $
  	\end{enumerate}

\item The least value of the function $ f(x) = 2\cos x + x $ in the closed interval $[0, \frac{\pi}{2}]$ is:

  	\begin{enumerate}
      	\item $ 2 $ 
      	\item $ \frac{\pi}{6} + \sqrt 3$
      	\item $ \frac{\pi}{2} $
	\item  The least value does not exist. 
  	\end{enumerate}

\item If $ x = a\sec \theta, y = b\tan \theta,$ then $ \frac{d^2y}{dx^2} $ at $ \theta = \frac{\pi}{2}$ is:
  
  	\begin{enumerate}
    	\item $ \frac{-3\sqrt 3b}{a^2} $
    	\item $ \frac{-2\sqrt 3b}{a} $
    	\item $ \frac{-3\sqrt 3b}{a} $
    	\item $ \frac{-b}{3 \sqrt 3a^2 }$
  	\end{enumerate}

\item The derivative of $ \sin^{-1} (2x \sqrt 1 - x^2) $ w.r.t $ \sin^{-1} x,  -\frac{1}{\sqrt 2 } < x < \frac{1}{\sqrt 2},$ is:
  
  	\begin{enumerate}
    	\item $ 2 $
    	\item $ \frac{\pi}{2} -2 $
    	\item $ \frac{\pi}{2} $
    	\item $ -2 $
  	\end{enumerate}

\item The point(s) on the curve  $ y = x^3 - 11x + 5 $ at which the tangent is $ y = x - 11 $ is/are:
  
  	\begin{enumerate}
    	\item $ (-2, 19)$
    	\item $ ( 2, -9)$
    	\item $ (\pm 2, 19) $
    	\item $ (-2 , 19) and (2, -9) $
  	\end{enumerate}

\item For which value of m is the line  $ y = mx + 1 $ a tangent to the curve $ y^2 = 4x $ ?
  
  	\begin{enumerate}
    	\item $ \frac{1}{2} $
    	\item $ 1 $
    	\item $ 2 $
    	\item $ 3 $
  	\end{enumerate}

\item The maximum value  of $ [x(x - 1) + 1]^\frac{1}{3},  0 \le x \le 1 $ is:
  
  	\begin{enumerate}
    	\item $0$
    	\item $\frac{1}{2}$
    	\item $1$
    	\item $\sqrt3  \frac{1}{3}$
  	\end{enumerate}

\end{enumerate}

\end{document}
